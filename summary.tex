\documentclass{oist}
\addbibresource{bibtex.bib}
\begin{document}

\pagestyle{empty}
\begin{titlepage}
    \AddToShipoutPicture*{	\put(-245,-180){%
            \parbox[b][\paperheight]{\paperwidth}{%
                \vfill
                \centering
                \includegraphics[width=0.95\paperwidth]{./images/symbol.png}%
                \vfill
            }}}
    \begin{center}
        \vfill
        {\large \scshape Università degli studi di Milano}\\[0.1cm]
        {\large \scshape Computer science department}\\[0.5cm]
        \rule{\textwidth}{1.5pt}\\[0cm]
        {\huge \bfseries  Implementazione di un algoritmo per il calcolo di refutazione per controllare la non-provabilità di una formula nella logica di G\"odel-Dummet \par \ }\\[-0.5cm]
        \rule{\textwidth}{1.5pt}\\[2.5cm]
        {\hfill \large Autore: \textbf{Samuele Orfei}} \\
        \vspace{0.1cm}
        {\hfill \large 922867} \\
        \vspace{0.5cm}
        {\hfill \large Relatore: \textbf{Camillo Fiorentini}} \\
        \vspace{1cm}
        \hfill  10/2023
    \end{center}
\end{titlepage}

\section*{Riassunto}

Le logiche di G\"odel-Dummett sono una famiglia infinita di logiche
intermedie proposizionali che estendono la logica intuizionista
proposizionale
e sono contenute nella logica classica proposizionale.
Dal punto di vista semantico, la logica di  G\"odel-Dummett di lunghezza
$k$,
che chiamiamo $\mathrm{GD}_k$,
\`e la logica caratterizzata dalla classe dei  modelli di Kripke lineari
di profondit\`a
al massimo $k$; la logica di  G\"odel-Dummett,  che chiamiamo $\mathrm{GD}$,
\`e caratterizzata dalla classe di tutti i  modelli di  Kripke lineari.
Si noti che $\mathrm{GD}_0\subset\mathrm{GD}_1
    \subset\mathrm{GD}_2\subset\dots\subset \mathrm{GD}$.
Questa famiglia di logiche \`e stata ampiamente studiata in letteratura
per le relazioni con le logiche
fuzzy \cite{Hajek:98} e perch\'e ammettono una particolare semantica computazionale
\cite{AscCiaGen:2017,Avron:91b}.
Sono noti vari calcoli e strategie di ricerche di prove per queste logiche.

In questo elaborato si presenta la implementazione in F\# del calcolo di
refutazione
presentato in \cite{cilc:2022}.
Pi\`u in dettaglio,
si tratta di un calcolo modulare in grado di dimostrare
la non-validit\`a di una formula in ciascuna delle logiche di
G\"odel-Dummett.
Data una logica $L$ e una formula $A$, se si costruisca una derivazione
di $A$ nel calcolo per $L$,
ne deriva che la formula $A$ non \`e valida nella logica $L$;
inoltre, dalla derivazione \`e possibile ottenere un contromodello per
$A$, ossia un modello
della logica $L$ in cui la formula $A$ \`e falsificata.
La particolarit\`a del calcolo \`e che supporta una strategia di ricerca
forward: prima vengono
generati tutti gli assiomi che possono contribuire alla costruzione
della derivazione;
successivamente, partendo dall'insieme degli assiomi, si applicano le
regole del calcolo, in tutti i modi possibili.

La procedura di ricerca termina quando si ottiene una derivazione  della
formula $A$ oppure tutte le possibili derivazioni  sono state costruite
e nessuna di queste \`e una derivazione di $A$; la terminazione \`e
garantita
dal fatto che sia  l'insieme iniziale degli assiomi che l'insieme delle
possibili applicazioni di regole sono finite.
Se la ricerca di una derivazione per $A$ fallisce,
si pu\`o concludere che la formula $A$ \`e valida nella logica $L$.
La prima sezione dell'elaborato si occupa di dare una definizione formale (matematica) del calcolo, di ogni suo passaggio e delle sue regole. Ci sono inoltre le definizioni formali di alcuni strumenti matematici utilizzati in alcuni passaggi del calcolo come le chiusure e le sottoformule.

Successivamente c'è una sezione dedicata allo pseudocodice delle principali funzioni implementate. Sono state selezionate le funzioni più significative e, nonostante l'implementazione sia in linguaggio F\# funzionale, ne è stata data la definizione mediante pseudocodice basato sul paradigma di programmazione procedurale (C-like). \\
Per ogni pseudocodice è stata fornita una breve spiegazione e le funzioni sono state divise a seconda dello scopo. Inoltre, per le funzioni di utilità più generiche, è stato fornito anche un esempio di input ed output.

La terza sezione contiene una descrizione riguardante il come è stata organizzata la soluzione e quali strumenti sono stati utilizzati per lo sviluppo. Il codice è stato scritto con la modularità in mente per poter, eventualmente, riutilizzare parti di codice anche al di fuori del progetto. Gli strumenti utilizzati sono:
\begin{itemize}
    \item \textbf{Github:} Per il versionamento del codice. 
    \item \textbf{.NET:} Come framework di sviluppo che fornisce sia il linguaggio F\# che le librerie contenti le strutture dati di base.
    \item \textbf{Visual Studio Code:} Come IDE di sviluppo. 
\end{itemize}

La quarta sezione contiene la descrizione dei vari moduli implementati con le relative funzioni e/o tipi. \\
Particolare attenzione è stata posta al tipo \textit{Formula} in quanto rappresenta il punto centrale di tutto il progetto. Tramite la \textit{Formula} è possibile rappresentare, sotto forma di albero, tutte le formule logiche e relativi operatori. \\
Quanto ai moduli invece, particolarmente importanti sono i moduli \textit{Expression} e \textit{Calculus} che contengono rispettivamente:
\begin{itemize}
    \item Le funzioni per l'estrapolazione di sottoformule o controllo di appartenenza a chiusure positive o negative
    \item Le funzioni per l'implementazione dell'algoritmo vero e proprio
\end{itemize}
Il flusso completo dell'algoritmo comprende:
\begin{itemize}
    \item Lettura della formula sotto forma di stringa da file di testo o da linea di comando
    \item Interpretazione della formula mediante un \textit{Interprete} composto da due moduli specifici: \textit{Lexer} per la costruzione dei \textit{Token} dal testo usando una grammatica specifica, \textit{Parser} che utilizza i \textit{Token} per costruire la rappresentazione interna della formula utilizzando il tipo \textit{Formula}
    \item Esecuzione dell'algoritmo vero e proprio, applicato sul risultato del \textit{Parser}
    \item Restituzione dei risultati e stampa sulla linea di comando di questi ultimi
\end{itemize}

La quinta sezione contiene la conclusione e la bibliografia. \\
La ricerca bibliografica ha riguardato principalmente i siti internet delle case madri dei software elencati sopra (.NET, Github, F\#) ed una pubblicazione su \textit{AIR} (Archivio Istituzionale della Ricerca) intitolato \textit{Forward refutation for G\"odel-Dummett Logics}. \\

\printbibliography
\addcontentsline{toc}{part}{Bibliografia}

\end{document}