%%%%%%%%%%%%%%%%%%%%%%%%%%%%%%%%%%%%%%%%%%%%%%%%%%%%%%%%%%%%%%%%%%%%
%% I, the copyright holder of this work, release this work into the
%% public domain. This applies worldwide. In some countries this may
%% not be legally possible; if so: I grant anyone the right to use
%% this work for any purpose, without any conditions, unless such
%% conditions are required by law.
%%%%%%%%%%%%%%%%%%%%%%%%%%%%%%%%%%%%%%%%%%%%%%%%%%%%%%%%%%%%%%%%%%%%

\documentclass{beamer}
\usetheme[university=bs,faculty=standard]{fibeamer}
\usepackage[utf8]{inputenc}
\usepackage[
  main=italian, %% By using `italian`as the main locale
                %% instead of `english`, you can typeset the
                %% presentation in Italian.
  english       %% The additional key allow foreign texts to be
]{babel}        %% typeset as follows:
%%
%%   \begin{otherlanguage}{italian}   ... \end{otherlanguage}
%%
%% These macros specify information about the presentation
\title{Implementazione di un algoritmo per il calcolo di refutazione per controllare la non-provabilità di una formula nella logica di G\"odel-Dummett} %% that will be typeset on the
%\subtitle{Presentation Subtitle} %% title page.
\author{Samuele Orfei}

%% These additional packages are used within the document:
\usepackage{ragged2e}  % `\justifying` text
\usepackage{booktabs}  % Tables
\usepackage{tabularx}
\usepackage{tikz}      % Diagrams
\usetikzlibrary{calc, shapes, backgrounds}
\usepackage{amsmath, amssymb}
\usepackage{url}       % `\url`s
\usepackage{listings}  % Code listings
\usepackage{graphicx}

\frenchspacing
\begin{document}
\frame[c]{\maketitle}

\begin{darkframes}

    \section{Presentazione}

    \subsection{Introduzione}
    \begin{frame}{Introduzione}

        \begin{block}{Logiche di G\"odel-Dummett}
            Famiglia di logiche dette \textit{intermedie}, caratterizzate da modelli lineari (di qualunque profondità).
        \end{block}

        Si svuole sviluppare un algoritmo che implementa un calcolo di refutazione per controllare la non-provabilità di una formula nella logica di \textit{G\"odel-Dummett}.\\

    \end{frame}

    \subsection{Requisiti Funzionali}
    \begin{frame}{Requisiti Funzionali}

        \textbf{Requisiti di base}
        \begin{itemize}
            \item Lettura della formula da standard input o file di testo
            \item Interpretazione della formula e costruzione della rappresentazione interna
            \item Applicazione del calcolo di refutazione e costruzione del contro-modello
            \item Stampa dei risultati sotto forma di applicazione di regole
        \end{itemize}

        \textbf{Requisiti sui tipi}
        \begin{itemize}
            \item Supporto degli operatori logici $\land$, $\lor$, $\supset$, $\not$
            \item Supporto di proposizioni atomiche e di costanti come $\bot$
            \item Supporto di formule innestate mediante interpretazione dei caratteri parentesi: tonda, quadra e graffa
            \item Supporto della lettura della formula sia sotto forma di stringa che di rappresentazione interna (tipo \textit{Formula})
        \end{itemize}

    \end{frame}

    \subsection{Requisiti di Integrazione}
    \begin{frame}{Requisiti di Integrazione}
        Le caratteristiche dei moduli e dell'algoritmo:
        \begin{itemize}
            \item Implementato in F\# compatibile con .NET 7.0
            \item Modularità: possibilità di riutilizzare il codice
            \item Controllo errori: in caso di errore, restituizione di errore con informazioni di traccia
            \item Tracciamento: le funzioni principali stampano messaggi sulle operazioni che stanno eseguendo
        \end{itemize}
    \end{frame}

    \subsection{Teoria}
    \begin{frame}{Definizioni formali}
        \framesubtitle{Definizioni}
        \begin{itemize}
            \item sottoformula sinistra o destra
            \item Chiusura positiva o negativa
            \item Sequente: Tripla di 3 insiemi ($\Lambda, \Gamma, \Delta$)
            \item Calcolo RGD: Precedimento ricorsivo di applicazione delle regole ai sequenti ed espansione del contro-modello
        \end{itemize}

        \framesubtitle{Regole algoritmo}
        \begin{itemize}
            \item Regola sinistra L->: Inserimento della sottoformula nell'insieme sinistro del sequente
            \item Regola destra R->: Inserimento della sottoformula nell'insieme destro del sequente
            \item Regola successore: Cambio di radice
            \item Condizioni di termininazione: Elementi finiti o sequenti saturi
        \end{itemize}
    \end{frame}

    \subsection{Strumenti}
    \begin{frame}{Strumenti}
        \framesubtitle{Strumenti essenziali}
        \begin{itemize}
            \item F\#
            \item .NET Framework
        \end{itemize}
    \end{frame}

    \begin{frame}{Strumenti}
        \framesubtitle{Altri strumenti importanti}
        \begin{itemize}
            \item Librerie utilizzate durante lo sviluppo (List, Set, Map)
            \item IDE e/o editor di testo (Visual Studio Code)
            \item Versionamento (Github)
        \end{itemize}
    \end{frame}

    \subsection{Struttura del progetto}
    \begin{frame}{Struttura del progetto}
        \begin{itemize}
            \item Modulo \textit{Lexer} che trasforma una formula sotto forma di stringa in un insieme di \textit{Token}
            \item Modulo \textit{Parser} che trasforma l'insieme di \textit{Token} nella rappresentazione interna della formula
            \item Modulo \textit{Expression} che racchiude le funzioni per la manipolazione, trasformazione o estrazione di dati da una data formula
            \item Modulo \textit{Calculus} che racchiude le funzioni che rappresentano l'algoritmo stesso
        \end{itemize}
    \end{frame}

    \subsection{Conclusioni}
    \begin{frame}{Conclusioni}
        \textbf{Stato del progetto}
        \begin{itemize}
            \item Algoritmo completato
        \end{itemize}
        \textbf{Prospettive future}
        \begin{itemize}
            \item Miglioramento dell'interprete ed espansione della grammatica supportata
            \item Ottimizzazione del calcolo stesso
        \end{itemize}
    \end{frame}

    \begin{frame}{Conclusioni}
        \textbf{Grazie per l'attenzione.}
    \end{frame}

\end{darkframes}

\end{document}