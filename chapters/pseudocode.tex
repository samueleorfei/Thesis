\providecommand{\main}{..}
\documentclass[\main/tesi.tex]{subfiles}
\begin{document}
\chapter{Pseudocodice}
\addcontentsline{toc}{chapter}{Pseudocodice}

In questa sezione verranno mostrate le funzioni principali per il funzionamento dell'algoritmo sotto forma di pseudocodice.

\section{Utilità}

Le funzioni di utilità utilizzate sono:
\begin{itemize}
    \item \textbf{atoms}
    \item \textbf{imps}
    \item \textbf{combinations}
\end{itemize}

\subsection{atoms}
Qui verrà presentato lo pseudocodice per la funzione \textit{atoms}, incaricata di estrapolare ricorsivamente tutte le possibili variabili proposizionali da una lista di formule. \\

\begin{algorithm}
    \caption{Pseudocodice per la funzione \textbf{atoms}}\label{alg:atoms}
    \begin{algorithmic}
        \Function{atoms}{set}
        \Function{single}{acc, f}
        \If{$f = Atom$}
        \State \textbf{return} concat(acc, [f])
        \ElsIf{$f = \bot$}
        \State \textbf{return} acc
        \Else
        \State \textbf{single}(acc, sottoFormule di f)
        \EndIf
        \State \textbf{return} acc
        \EndFunction
        \ForAll{$f \in set$}
        \State $f \gets \textbf{single}([], f)$
        \EndFor
        \State \textbf{return} set
        \EndFunction
    \end{algorithmic}
\end{algorithm}

\subsection{imps}
Qui verrà presentato lo pseudocodice per la funzione \textit{imps}, incaricata di estrapolare ricorsivamente tutte le possibili formula dalla forma $A \Rightarrow B$ da una lista di formule. \\

\begin{algorithm}
    \caption{Pseudocodice per la funzione \textbf{imps}}\label{alg:imps}
    \begin{algorithmic}
        \Require $set \neq \emptyset$
        \Ensure res senza duplicati
        \State $res = \emptyset$
        \State $i = 0$
        \While{$set \neq \emptyset$}
        \State $x = set[i]$
        \If{$x$ is $\neg$ or $\Rightarrow$}
        aggiungi x a res
        \State $i = i + 1$
        \EndIf
        \EndWhile
    \end{algorithmic}
\end{algorithm}

\subsection{combinations}
Qui verrà presentato lo pseudocodice per la funzione \textit{combinations}, incaricata di creare tutte le combinazioni possibili degli elementi di una lista di formule e di restituirle sotto forma di lista di liste di formule. \\

\begin{algorithm}
    \caption{Pseudocodice per la funzione \textbf{combinations}}\label{alg:combinations}
    \begin{algorithmic}
        \Function{combinations}{lst}
        \Function{comb}{accLst, elemLst}
        \Repeat
        \State $top \gets \textbf{pop}(elemLst)$
        \State $res \gets $ concatenazione di $top$ davanti ad ogni elemento di $concat(accList, accList)$
        \State $next \gets \textbf{concat}(top, res)$
        \State comb(next, elemLst)
        \Until{$elemLst$ is empty}
        \State \textbf{return} accList
        \EndFunction
        \State \textbf{return} comb([], lst)
        \EndFunction
    \end{algorithmic}
\end{algorithm}

\section{Calculus}
Il calcolo di refutazione è composto da 3 funzioni principali:
\begin{itemize}
    \item \textbf{generateAxioms}
    \item \textbf{execute}
    \item \textbf{printResult}
\end{itemize}

\subsection{generateAxioms}
Qui verrà presentato lo pseudocodice per la funzione \textit{generateAxioms}, incaricata di creare tutti gli assiomi possibili che avranno la forma di sequenti. \\

\begin{algorithm}
    \caption{Pseudocodice per la funzione \textbf{generateAxioms}}\label{alg:generateAxioms}
    \begin{algorithmic}
        \Function{generateAxioms}{subFormulas}
        \State $atomSF \gets \textbf{atoms}(subFormulas)$
        \State $combs \gets \textbf{combinations}(atomSF)$
        \ForAll{$x \in combs$}
        \State $x \gets (\{x\}, \emptyset, \{\bot, \{atomSF\} - \{x\}\})$
        \EndFor
        \State \textbf{return} \textbf{concat}(combs, [($\emptyset, \emptyset, \{atomSF, \bot\}$)])
        \EndFunction
    \end{algorithmic}
\end{algorithm}

\subsection{execute}
Qui verrà presentato lo pseudocodice per la funzione \textit{execute}, che rappresenta il vero e proprio calcolo di refutazione. Una volta creati gli assiomi iniziali, applica tutte le regole a tutti i sequenti generati negli step successivi fino ad ottenere un sequente che rispetta le condizioni di interruzione. \\

\begin{comment}
\begin{algorithm}
    \caption{Pseudocodice per la funzione \textbf{proof}}\label{alg:proof}
    \begin{algorithmic}
        \Function{proof}{$G, \Gamma, \Lambda, \Delta, sl, sf, k, history$}
        \While{($G \notin Cl^{-}(\Delta \cup \Lambda)) \lor (sl \land sr \neq \emptyset)$}
        \State $leftItems \gets $ elementi in sl che rispettano $L\Rightarrow$
        \State $rightItems \gets $ elementi in sr che rispettano $R\Rightarrow$
        \If{$leftItems \land rightItems = \emptyset$}
        \State $\Lambda^{'} = \Gamma^{at} \cap \Delta^{at}$
        \If{$\Lambda^{'} = \Lambda$}
        \State \textbf{return} null
        \Else
        \State $result \gets null$
        \ForAll{$comb \in \textbf{combinations(\Lambda^{'})}$}
        \State $nG \gets \Gamma - comb$
        \State $nD \gets \Delta \cup \Lambda$
        \State $nK \gets k + 1$
        \State $nH \gets \textbf{concat}(history, [(Succ, bG, comb, nD, nK)])$
        \State $result \gets \textbf{proof}(G, nG, nD, comb, sl, sr, nK, nH)$

        \If{$result \neq null$}
        \State \textbf{return} result
        \Else
        \State \textbf{continue}
        \EndIf
        \EndFor
        \EndIf
        \Else
        \State $lF \gets \textbf{pop}(leftItems)$
        \State $rF \gets \textbf{pop}(rightItems)$
        \State $nD \gets \Delta, rF$
        \State $nG \gets \Gamma, lF$
        \State $lH \gets \textbf{concat}(history, [(Left, nG, \Lambda, \Delta, k)])$
        \State $rH \gets \textbf{concat}(history, [(Right, \Gamma, \Lambda, nD, k)])$

        \State{$lR \gets \textbf{proof}(G, nG, \Lambda, \Delta, \{sl - lF\}, sr, k, lH)$}
        \State{$rR \gets \textbf{proof}(G, \Gamma, \Lambda, nD, sl, \{sr - rF\}, k, rH)$}

        \If{$lR \neq null \land rR \neq null$}
        \State \textbf{return} lR
        \ElsIf{$lR = null \land rR \neq null$}
        \State \textbf{return} rR
        \ElsIf{$lR \neq null \land rR = null$}
        \State \textbf{return} lR
        \Else
        \State \textbf{return} null
        \EndIf
        \EndIf
        \EndWhile
        \EndFunction
    \end{algorithmic}
\end{algorithm}
\end{comment}

\begin{algorithm}
    \caption{Pseudocodice per la funzione \textbf{execute}}\label{alg:execute}
    \begin{algorithmic}
        \Function{execute}{goal}
        \State $(sl, sr) \gets \textbf{subFormulas}(goal)$
        \State $sf \gets \textbf{concat}(sl, sr)$
        \State $impSL \gets \textbf{imps}(sl)$
        \State $impSR \gets \textbf{imps}(sr)$
        \State $axioms \gets \textbf{generateAxioms}(sf)$
        \State $result = null$
        \ForAll{$(Rule, \Gamma, \Lambda, \Delta, k) \in axioms$}
        \If{($\Gamma^{at} \cup \Delta^{at} = Sf^{at}(G)) \land (\Gamma^{at} \cap \Delta^{at} = \emptyset)$}
        \State $result = \textbf{proof}(goal, \Gamma, \Delta, \Lambda, sl, sr, 0, [(Ax, \Gamma, \Lambda, \Delta, 0)])$
        \Else
        \State \textbf{continue}
        \EndIf

        \If{$result \neq null$}
        \State \textbf{break}
        \EndIf
        \EndFor
        \State \textbf{return} result
        \EndFunction
    \end{algorithmic}
\end{algorithm}

\subsection{printResult}
Qui verrà presentato lo pseudocodice per la funzione \textit{printResult}, che prende in input il risultato della funzione \textit{execute} e stampa sul terminale gli step che hanno portato alla costruzione del contro-modello indentati in maniera comprensibile. \\

\begin{algorithm}
    \caption{Pseudocodice per la funzione \textbf{printResult}}\label{alg:printResult}
    \begin{algorithmic}
        \Procedure{printResult}{results}
        \ForAll{$(Rule, \Gamma, \Lambda, \Delta, k) \in results$}
        \If{$Rule = Ax$}
        \State printf "$K = k, Ax$"
        \ElsIf{$Rule = Left$}
        \State printf "$K = k, L\Rightarrow$"
        \ElsIf{$Rule = Right$}
        \State printf "$K = k, R\Rightarrow$"
        \ElsIf{$Rule = Succ$}
        \State printf "$K = k, Succ$"
        \EndIf
        \State printf "$\Gamma \not\Rightarrow \Lambda$; $\Delta$"
        \EndFor
        \EndProcedure
    \end{algorithmic}
\end{algorithm}

\end{document}