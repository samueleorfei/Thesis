\providecommand{\main}{..}
\documentclass[\main/tesi.tex]{subfiles}
\begin{document}
\chapter{Preliminari}
\addcontentsline{toc}{chapter}{Preliminari}

Le formule, denotate da lettere latine maiuscole, sono costruite da un insieme infinito di variabili proposizionali \(V = \{p,q,p1,p2,...\}\), dalla costante \(\bot\) e dai connettivi \(\land, \lor, \Rightarrow\). \\
E' importante sottolineare il fatto che nella logica di G\"odel-Dummett, l'espressione \(\neg A\) rappresenta la forma abbreviata per \(A \Rightarrow \bot\). \\

L'implementazione dell'algoritmo fa uso di alcune nozioni matematiche per la manipolazione e l'estrapolazione di informazioni da una \textit{formula}, quali: \\
\begin{itemize}
    \item Sotto-formula
    \item Sotto-formula sinistra
    \item Sotto-formula destra
    \item Chiusura positiva
    \item Chiusura negativa
\end{itemize}

\section{Sottoformule}

Sia G una formula; Sf(G) è l'insieme di tutte le sottoformule di G, inclusa G stessa.

Una sottoformula è una formula contenuta in G (Es. Se \(G = A \land B\), le sotto-formule sono: A, B e G stessa). \\
Le sotto-formule possono dividersi in altre 2 categorie di formule:
\begin{itemize}
    \item Negative (o sinistre)
    \item Positive (o destre)
\end{itemize}

Con il termine SL(G) denotiamo il sotto-insieme di formule negative estratto da Sf(G), mentre con SR(G) denotiamo il sotto-insieme di formule positive. \\
Formalmente, SL(G) ed SR(G) sono i più piccoli sotto-insiemi di Sf(G) tali che:
\begin{itemize}
    \item \(G \in SR(G)\)
    \item \(A \odot B \in SX(G) \Rightarrow \{A, B\} \subseteq SX(G)\), dove \(\odot \in \{\land, \lor\}\), e \(SX \in \{SL, SR\}\)
    \item \(A \Rightarrow B \in SL(G) \Rightarrow B \in SL(G) \land A \in SR(G)\)
    \item \(A \Rightarrow B \in SR(G) \Rightarrow B \in SR(G) \land A \in SL(G)\)
\end{itemize}

Per \(SX \in \{SL, SR\}\) abbiamo:
\begin{itemize}
    \item $SX^{at}(G) = SX(G) \cap V$
    \item $SX^{at, \Rightarrow}(G) = SX^{at}(G) \cup SX^{\Rightarrow}(G)$
    \item $SX^{\Rightarrow}(G) = SX(G) \cap L^{\Rightarrow}$
    \item $Sf^{at}(G) = SL^{at}(G) \cup SR^{at}(G)$
\end{itemize}

\section{Chiusure}

Sia S un insieme di formule; Consideriamo le formule P ed N definite dalla seguente grammatica, dove \(A \in S\) ed F è una formula qualsiasi:
\begin{itemize}
    \item \(P ::= A|P \land P|F \lor P|P \lor F|F \Rightarrow P\)
    \item \(N ::= A|N \lor N|F \land P|P \land F\)
\end{itemize}
La \textit{chiusura positive} di S, denotata da $Cl^+$(S), è il più piccolo insieme contenente le formule descritte dalla grammatica P;
La \textit{chiusura negativa} di S, denotata da $Cl^-$(S), è il più piccolo insieme contenente le formule descritte dalla grammatica N. \\

Nel caso una formula sia composta da più sotto-formule, va eseguita una ricerca ricorsiva per capire se è contenuta nella chiusura positiva o negativa. \\

\end{document}