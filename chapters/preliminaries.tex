\providecommand{\main}{..}
\documentclass[\main/tesi.tex]{subfiles}
\begin{document}
\chapter{Preliminari}
\addcontentsline{toc}{chapter}{Preliminari}

Le formule, denotate da lettere latine maiuscole, sono costruite da un insieme infinito di variabili proposizionali $V = \{p,q,p_1,p_2,...\}$, dalla costante $\bot$ e dai connettivi $\land, \lor, \supset$. \\
E' importante sottolineare il fatto che nella logica di G\"odel-Dummett, l'espressione $\neg A$ rappresenta la forma abbreviata per $A \supset \bot$. \\

L'implementazione dell'algoritmo fa uso di alcune nozioni matematiche per la manipolazione e l'estrapolazione di informazioni da una \textit{formula}, quali: \\
\begin{itemize}
    \item sottoformula
    \item sottoformula sinistra
    \item sottoformula destra
    \item Chiusura positiva
    \item Chiusura negativa
\end{itemize}

\newpage

\section{Sottoformule}

Sia G una formula; $Sf(G)$ è l'insieme di tutte le sottoformule di $G$, inclusa $G$ stessa.

Una sottoformula è una formula contenuta in G (Es. Se $G = A \land B$, le sottoformule sono: $A$, $B$ e $G$ stessa). \\
Le sottoformule possono dividersi in altre 2 categorie di formule:
\begin{itemize}
    \item Negative (o sinistre)
    \item Positive (o destre)
\end{itemize}

Con il termine $SL(G)$ denotiamo il sottoinsieme di formule negative estratto da $Sf(G)$, mentre con $SR(G)$ denotiamo il sottoinsieme di formule positive. \\
Formalmente, $SL(G)$ ed $SR(G)$ sono i più piccoli sottoinsiemi di $Sf(G)$ tali che:
\begin{itemize}
    \item $G \in SR(G)$
    \item $A \odot B \in SX(G) \supset \{A, B\} \subseteq SX(G)$, dove $\odot \in \{\land, \lor\}$, e $SX \in \{SL, SR\}$
    \item $A \supset B \in SL(G) \supset B \in SL(G) \land A \in SR(G)$
    \item $A \supset B \in SR(G) \supset B \in SR(G) \land A \in SL(G)$
\end{itemize}

Per $SX \in \{SL, SR\}$ abbiamo:
\begin{itemize}
    \item $SX^{at}(G) = SX(G) \cap V$
    \item $SX^{at, \supset}(G) = SX^{at}(G) \cup SX^{\supset}(G)$
    \item $SX^{\supset}(G) = SX(G) \cap L^{\supset}$
    \item $Sf^{at}(G) = SL^{at}(G) \cup SR^{at}(G)$
\end{itemize}

\newpage

\section{Chiusure}

Sia $S$ un insieme di formule; Consideriamo le formule P ed N definite dalla seguente grammatica, dove $A \in S$ ed F è una formula qualsiasi:
\begin{itemize}
    \item $P ::= A|P \land P|F \lor P|P \lor F|F \supset P$
    \item $N ::= A|N \lor N|F \land P|P \land F$
\end{itemize}
La \textit{chiusura positive} di $S$, denotata da $Cl^+(S)$, è il più piccolo insieme contenente le formule descritte dalla grammatica $P$;
La \textit{chiusura negativa} di $S$, denotata da $Cl^-(S)$, è il più piccolo insieme contenente le formule descritte dalla grammatica $N$. \\

Nel caso una formula sia composta da più sottoformule, va eseguita una ricerca ricorsiva per capire se è contenuta nella chiusura positiva o negativa. \\

\section{Calcolo}

\subsubsection{Definizione formale}
Sia $G$ la formula obiettivo; Il calcolo $RGD(G)$ è un procedimento per inferire la non-provabilità di $G$ in $GD_k$ ed è pensato per supportare una ricerca in avanti di una refutazione.\\
Il calcolo agisce sui $RGD(G)$-sequenti che hanno la forma $\Gamma \not\Rightarrow_k \Lambda; \Delta$, dove:
\begin{itemize}
    \item Se $k = 0$, allora $\Lambda = \emptyset$
    \item Se $k \geq 0$, allora $\Gamma \subseteq SL^{at,\supset}(G)$, $\Lambda \subseteq SL^{at}(G) \cap SR^{at}(G)$, $\Delta \subseteq SR^{at,\supset}(G)$
\end{itemize}
Il rango di $\sigma = \Gamma \not\Rightarrow_k \Lambda; \Delta$, identificato da Rn($\sigma$), è k. Un sequente è un'entità della logica che permette di esprimere legami tra asserzioni.

Il calcolo è una procedura che si compone di più fasi. Dato un obiettivo $G$, i passaggi che deve eseguire sono:
\begin{itemize}
    \item Creazione degli assiomi di partenza che comporranno un sequente del tipo: $\Gamma^{at} \not\Rightarrow_0 \bullet; \Delta^{at}, \bot$ (gli assiomi sono le regole senza premesse)
    \item Applicazione ciclica di tutte le possibili regole ai sequenti collezionati nei passaggi precedenti. Le regole possono essere di 3 tipi:
          \begin{itemize}
              \item Regola sinistra ($L\supset$): Viene aggiunta una nuova formula del tipo $A \supset B$ nella parte sinistra del sequente, che corrisponde all'insieme $\Gamma$. La formula scelta, deve rispettare determinate condizioni per poter essere aggiunta all'insieme $\Gamma$, ossia:
                    \begin{itemize}
                        \item Caso $\Lambda = \emptyset$:
                              \begin{itemize}
                                  \item La formula NON deve essere compresa nell'insieme unione tra $\Gamma$ e $\Delta$ ($A \supset B \notin \Gamma \cup \Delta$)
                                  \item Il primo operatore dell'implicazione deve appartenere alla chiusura negativa di $\Delta$ ($A \in Cl^{-}(\Delta)$)
                              \end{itemize}
                        \item Altrimenti:
                              \begin{itemize}
                                  \item La formula NON deve essere compresa nell'insieme unione tra $\Gamma$ e $\Delta$ ($A \supset B \notin \Gamma \cup \Delta$)
                                  \item Il primo operatore dell'implicazione deve appartenere alla chiusura negativa dell'unione tra $\Delta$ e $\Lambda$ ($A \in Cl^{-}(\Delta \cup \Lambda)$)
                                  \item Il secondo operatore dell'implicazione deve appartenere alla chiusura positiva dell'unione tra $\Gamma$ e $\Lambda$ ($B \in Cl^{+}(\Gamma \cup \Lambda)$)
                              \end{itemize}
                    \end{itemize}
              \item Regola destra ($R\supset$): Viene aggiunta una nuova formula del tipo $A \supset B$ nella parte destra del sequente, che corrisponde all'insieme $\Delta$. La formula scelta, deve rispettare determinate condizioni per poter essere aggiunta all'insieme $\Delta$, ossia:
                    \begin{itemize}
                        \item La formula NON deve essere compresa nell'insieme unione tra $\Delta$ e $\Delta$ ($A \supset B \notin \Delta \cup \Delta$)
                        \item Il primo operatore dell'implicazione deve appartenere alla chiusura positiva dell'insieme $\Gamma$ ($A \in Cl^{+}(\Gamma)$)
                        \item Il secondo operatore dell'implicazione deve appartenere alla chiusura negativa dell'unione tra $\Delta$ e $\Lambda$ ($B \in Cl^{-}(\Delta \cup \Lambda)$)
                    \end{itemize}
              \item Regola "successore": In termini di costruzione del contro-modello, questa regola corrisponde all'espansione di un modello, tramite aggiunta di una nuova radice al posto di quella corrente. A livello di sequenti, si tratta di generare tutti i possibili insiemi $\Lambda^{'}$ che rispettano una certa regola e di verificare, reiterando l'algoritmo per ognuno di essi, se ce n'è uno che porta alla soluzione. Per ogni $\Lambda^{'}$, si eliminano da $\Gamma$ gli elementi in comune e si aggiungono in $\Lambda$. Dopodichè si aggiunge l'intero $\Lambda$ originale a $\Delta$. I criteri con cui vengono generati i $\Lambda^{'}$ sono:
                    \begin{itemize}
                        \item L'insieme deve essere sottoinsieme/uguale all'intersezione di $\Gamma$ e l'insieme delle variabili proposizionali $V$ e non deve essere vuoto ($\emptyset \subset \Lambda^{'} \subseteq \Gamma \cap V$).
                        \item Questa regola va applicata SOLO quando il sequente è saturo (ossia quando non è possibile aggiungere nuove formule nè a sinistra nè a destra)
                    \end{itemize}
          \end{itemize}
\end{itemize}

Le condizioni per la terminazione sono due:
\begin{itemize}
    \item Gli insiemi di sottoformule sinistre e destre dell'obiettivo hanno finito gli elementi (e quindi sarebbe impossibile trovare nuove implicazioni da aggiungere)
    \item Quando si ottiene un sequente che rispetta la seguente regola: $G \in Cl^{-}(\Delta \cup \Lambda)$
\end{itemize}

\end{document}