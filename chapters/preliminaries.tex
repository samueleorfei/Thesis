\providecommand{\main}{..}
\documentclass[\main/tesi.tex]{subfiles}
\begin{document}
\chapter{Preliminari}
\addcontentsline{toc}{chapter}{Preliminari}

In questa sezione verranno discusse le definizioni formali delle principali nozioni logiche utilizzate per l'implementazione dell'algoritmo.

\section{Definizione}

Le formule, denotate da lettere latine maiuscole, sono costruite da un insieme infinito di variabili proposizionali $V = \{p,q,p_1,p_2,...\}$, dalla costante $\bot$ e dai connettivi $\land, \lor, \supset$. \\
Nel seguito viene anche usato il connettivo definito $\neg$: la formula $\neg A$ è una abbreviazione per $A \supset \bot$. \\

L'implementazione dell'algoritmo fa uso di alcune nozioni matematiche per la manipolazione e l'estrapolazione di informazioni da una \textit{formula}, quali: \\
\begin{itemize}
    \item Sottoformula.
    \item Sottoformula sinistra.
    \item Sottoformula destra.
    \item Chiusura positiva.
    \item Chiusura negativa.
\end{itemize}

\newpage

\section{Sottoformule}

Sia G una formula; $Sf(G)$ è l'insieme di tutte le sottoformule di $G$, inclusa $G$ stessa.
Formalmente:
\begin{itemize}
    \item $G \in Sf(G)$
    \item Se $A \land B \in Sf(G)$, allora $A \in Sf(G)$ e $B \in Sf(G)$
    \item Se $A \lor B \in Sf(G)$, allora $A \in Sf(G)$ e $B \in Sf(G)$
    \item Se $A \supset B \in Sf(G)$, allora $A \in Sf(G)$ e $B \in Sf(G)$
\end{itemize}

Le sottoformule possono dividersi in altre 2 categorie di formule:
\begin{itemize}
    \item Negative (o sinistre)
    \item Positive (o destre)
\end{itemize}

Con il termine $SL(G)$ denotiamo il sottoinsieme di formule negative estratto da $Sf(G)$, mentre con $SR(G)$ denotiamo il sottoinsieme di formule positive (\cite{TroSch:00}). \\
Formalmente, $SL(G)$ ed $SR(G)$ sono i più piccoli sottoinsiemi di $Sf(G)$ tali che:
\begin{itemize}
    \item $G \in SR(G)$
    \item $A \odot B \in SX(G) \Rightarrow \{A, B\} \subseteq SX(G)$, dove $\odot \in \{\land, \lor\}$, e $SX \in \{SL, SR\}$
    \item $A \supset B \in SL(G) \Rightarrow B \in SL(G) \land A \in SR(G)$
    \item $A \supset B \in SR(G) \Rightarrow B \in SR(G) \land A \in SL(G)$
\end{itemize}

Per $SX \in \{SL, SR\}$ abbiamo:
\begin{itemize}
    \item $SX^{at}(G) = SX(G) \cap V$
    \item $SX^{at, \supset}(G) = SX^{at}(G) \cup SX^{\supset}(G)$
    \item $SX^{\supset}(G) = SX(G) \cap L^{\supset}$
    \item $Sf^{at}(G) = SL^{at}(G) \cup SR^{at}(G)$
\end{itemize}

\newpage

\section{Chiusure}

Sia $S$ un insieme di formule; Consideriamo le formule P ed N definite dalla seguente grammatica, dove $A \in S$ ed F è una formula qualsiasi:
\begin{itemize}
    \item $P ::= A|P \land P|F \lor P|P \lor F|F \supset P$
    \item $N ::= A|N \lor N|F \land P|P \land F$
\end{itemize}
La \textit{chiusura positive} di $S$, denotata da $Cl^+(S)$, è il più piccolo insieme contenente le formule descritte dalla grammatica $P$;
La \textit{chiusura negativa} di $S$, denotata da $Cl^-(S)$, è il più piccolo insieme contenente le formule descritte dalla grammatica $N$. \\

Nel caso una formula sia composta da più sottoformule, va eseguita una ricerca ricorsiva per capire se è contenuta nella chiusura positiva o negativa. \\

\section{Le logiche $GD_k$ e $GD$}

consideriamo le logiche di G\"odel-Dummett $GD_k$ ($k \geq 0$) e $GD$ definite nel seguente modo (\cite{ChaZak:97}):
\begin{itemize}
    \item $GD_k$ è l'insieme di formule valide nei modelli lineari $K$, tali che $h(K) \leq k$ (\cite[pp.\ 2--3]{DBLP:conf/cilc/Fiorentini022}).
    \item $GD = \bigcap_{k \geq 0}GD_k$
\end{itemize}
Va sottolineato il fatto che $IPL \subset GD \subset \dots \subset GD_2 \subset GD_1 \subset GD_0 = CPL$, dove $CPL$ è l'insieme delle formule valide nella logica classica.

\section{Calcolo}

\subsubsection{Definizione formale}
Sia $G$ la formula obiettivo; Il calcolo $RGD(G)$ è un procedimento per inferire la non-provabilità di $G$ in $GD_k$ ed è pensato per supportare una ricerca in avanti di una refutazione.\\
Il calcolo agisce sui $RGD(G)$-sequenti che hanno la forma $\Gamma \not\Rightarrow_k \Lambda; \Delta$, dove:
\begin{itemize}
    \item Se $k = 0$, allora $\Lambda = \emptyset$
    \item Se $k \geq 0$, allora $\Gamma \subseteq SL^{at,\supset}(G)$, $\Lambda \subseteq SL^{at}(G) \cap SR^{at}(G)$, $\Delta \subseteq SR^{at,\supset}(G)$
\end{itemize}
Il rango di $\sigma = \Gamma \not\Rightarrow_k \Lambda; \Delta$, identificato da Rn($\sigma$), è k. Un sequente è un termine della logica che permette di esprimere legami tra asserzioni.

Il calcolo è una procedura che si compone di più fasi. Dato un obiettivo $G$, i passaggi che deve eseguire sono:
\begin{itemize}
    \item Creazione degli assiomi di partenza che comporranno un sequente del tipo: $\Gamma^{at} \not\Rightarrow_0 \bullet; \Delta^{at}, \bot$ (gli assiomi sono le regole senza premesse)
    \item Applicazione ciclica di tutte le possibili regole ai sequenti collezionati nei passaggi precedenti. Le regole possono essere di 3 tipi:
          \begin{itemize}
              \item Regola sinistra ($L\supset$): Viene aggiunta una nuova formula del tipo $A \supset B$ nella parte sinistra del sequente, che corrisponde all'insieme $\Gamma$. La formula scelta, deve rispettare determinate condizioni per poter essere aggiunta all'insieme $\Gamma$, ossia:
                    \begin{itemize}
                        \item Caso $\Lambda = \emptyset$:
                              \begin{itemize}
                                  \item La formula NON deve essere compresa nell'insieme unione tra $\Gamma$ e $\Delta$ ($A \supset B \notin \Gamma \cup \Delta$)
                                  \item L'antecedente dell'implicazione deve appartenere alla chiusura negativa di $\Delta$ ($A \in Cl^{-}(\Delta)$)
                              \end{itemize}
                        \item Altrimenti:
                              \begin{itemize}
                                  \item La formula NON deve essere compresa nell'insieme unione tra $\Gamma$ e $\Delta$ ($A \supset B \notin \Gamma \cup \Delta$)
                                  \item L'antecedente dell'implicazione deve appartenere alla chiusura negativa dell'unione tra $\Delta$ e $\Lambda$ ($A \in Cl^{-}(\Delta \cup \Lambda)$)
                                  \item Il conseguente dell'implicazione deve appartenere alla chiusura positiva dell'unione tra $\Gamma$ e $\Lambda$ ($B \in Cl^{+}(\Gamma \cup \Lambda)$)
                              \end{itemize}
                    \end{itemize}
              \item Regola destra ($R\supset$): Viene aggiunta una nuova formula del tipo $A \supset B$ nella parte destra del sequente, che corrisponde all'insieme $\Delta$. La formula scelta, deve rispettare determinate condizioni per poter essere aggiunta all'insieme $\Delta$, ossia:
                    \begin{itemize}
                        \item La formula NON deve essere compresa nell'insieme unione tra $\Delta$ e $\Delta$ ($A \supset B \notin \Delta \cup \Delta$)
                        \item L'antecedente dell'implicazione deve appartenere alla chiusura positiva dell'insieme $\Gamma$ ($A \in Cl^{+}(\Gamma)$)
                        \item Il conseguente dell'implicazione deve appartenere alla chiusura negativa dell'unione tra $\Delta$ e $\Lambda$ ($B \in Cl^{-}(\Delta \cup \Lambda)$)
                    \end{itemize}
              \item Regola "successore": In termini di costruzione del contro-modello, questa regola corrisponde all'espansione di un modello, tramite aggiunta di una nuova radice al posto di quella corrente. A livello di sequenti, si tratta di generare tutti i possibili insiemi $\Lambda^{'}$ che rispettano una certa regola e di verificare, reiterando l'algoritmo per ognuno di essi, se ce n'è uno che porta alla soluzione. Per ogni $\Lambda^{'}$, si eliminano da $\Gamma$ gli elementi in comune e si aggiungono in $\Lambda$. Dopodichè si aggiunge l'intero $\Lambda$ originale a $\Delta$. I criteri con cui vengono generati i $\Lambda^{'}$ sono:
                    \begin{itemize}
                        \item L'insieme deve essere sottoinsieme/uguale all'intersezione di $\Gamma$ e l'insieme delle variabili proposizionali $V$ e non deve essere vuoto ($\emptyset \subset \Lambda^{'} \subseteq \Gamma \cap V$).
                        \item Questa regola va applicata SOLO quando il sequente è saturo (ossia quando non è possibile aggiungere nuove formule nè a sinistra nè a destra)
                    \end{itemize}
          \end{itemize}
\end{itemize}

Le condizioni per la terminazione sono due:
\begin{itemize}
    \item Gli insiemi di sottoformule sinistre e destre dell'obiettivo hanno finito gli elementi (e quindi sarebbe impossibile trovare nuove implicazioni da aggiungere)
    \item Quando si ottiene un sequente che rispetta la seguente regola: $G \in Cl^{-}(\Delta \cup \Lambda)$
\end{itemize}

Di seguito verranno mostrate le formalizzazioni delle regole del calcolo spiegate sopra.

\begin{figure}[h]
    \[
        \begin{array}{c}
            %% AXIOM
            \AXC{}
            \RightLabel{$\mathrm{Ax}$}
            \UIC{$\seqfrd{0}{\Gat}{}{\Dat,\bot}$}
            \DP
            \qquad
            \begin{minipage}{15em}
                $\Gat\cup\Dat\,=\,\Sfat{G}$
                \\[1ex]
                $\Gat\cap\Dat=\emptyset$
            \end{minipage}
            \\[9ex]
            %%% L IMP  0   
            \AXC{$\seqfrd{0}{\G}{}{\D}$}
            \RightLabel{$L\imp$}
            \UIC{$\seqfrd{0}{A\imp B,\G}{}{\D}$}
            \DP
            \quad
            \begin{minipage}{7em}
                $A\imp B\not\in\G\cup\D$
                \\[1ex]
                $A\in\Clom{\D}$
            \end{minipage}
            \hspace{1em}
            %%% L IMP  1
            \AXC{$\seqfrd{k}{\G}{\Lambda}{\D}$}
            \RightLabel{$L\imp$}
            \UIC{$\seqfrd{k}{A\imp B,\G}{\Lambda}{\D}$}
            \DP
            \quad
            \begin{minipage}{10em}
                $A\imp B\not\in\G\cup\D$
                \\[1ex]
                $A\in\Clom{\D\cup\Lambda}$
                \\[1ex]
                $B\in\Clop{\G\cup\Lambda}$
            \end{minipage}
            \\[9ex]
            %%% R IMP  
            \AXC{$\seqfrd{k}{\G}{\Lambda} {\D}$}
            \RightLabel{$R\imp$}
            \UIC{$\seqfrd{k}{\G}{\Lambda}{\D,A\imp B}$}
            \DP
            \quad
            \begin{minipage}{8em}
                $A\imp B\not\in\Delta\cup\D$
                \\[1ex]
                $A\in\Clop{\G}$
                \\[1ex]
                $B\in\Clom{\D\cup\Lambda}$
            \end{minipage}
            \hspace{2em}
            \\[9ex]
            %%% SUCC
            \AXC{$\seqfrd{k}{\G}{\Lambda}{\D}$}
            \RightLabel{$\mathrm{Succ}$}
            \UIC{$\seqfrd{k+1}{\G\setminus\Lambda'}{\Lambda'}{\D,\Lambda}$}
            \DP
            \quad
            \begin{minipage}{15em}
                $\seqfrd{k}{\G}{\Lambda}{\D}$ \'e saturo
                \\[1ex]
                % $\Lambda' \subseteq\PV$           \\[1ex]
                $\emptyset\subset \Lambda' \subseteq G \cap PV $
            \end{minipage}
        \end{array}
    \]
    \caption{Il calcolo di refutazione ${\bf RGD}(G)$.}
    \label{fig:frdum}
\end{figure}

\newpage

\begin{example}\label{ex:ex1}
    Consideriamo la seguente formula $G$:
    \[
        \begin{array}{l}
            G \;=\;
            A\lor (p\imp r) \lor B \lor (C\imp (p\lor \neg p))
            \\[1.5ex]
            A\;=\,\neg(q\land r)
            \qquad
            B \;=\;(\neg \neg p \land (p\imp q))\imp q
            \qquad
            C\;=\;B\imp (\neg\neg p \land q)
        \end{array}
    \]
    Cerchiamo una derivazione ${\bf RGD}(G)$ costruendo un database dei sequenti secondo le indicazioni
    descritte in ~\cite{VoronkovHAR:01}: cominciamo con l'inserimento di tutti gli assiomi; poi entriamo in un ciclo dove, ad ogni iterazione,
    applichiamo tutte le possibile regole ai sequenti collezionati nei passaggi precedenti. Il ciclo termine quanto un sequente del tipo $\seqfrd{k}{\Gamma}{\Lambda}{\D}$ con
    $G\in\Clom{\D\cup\Lambda}$ viene generato, oppure quando nessun altro sequente pu\'o essere aggiunto al database (database saturo).
    Fig.~\ref{fig:ex1} mostra il frammento di database contenente i sequenti necessari per ottenere la derivazione ${\bf RGD}(G)$ di $G$. Nell'esempio,
    identifichiamo con $\s_{(j)}$ il sequente alla linea~$(j)$ della
    Fig.~\ref{fig:ex1}. Come si pu\'o notare, il sequente $\s_{\eqref{ex1:2}}$
    viene ottenuto applicando la regola $R\imp$ al sequente
    $\s_{\eqref{ex1:1}}$, cio\'e,:
    \[
        \AXC{$\seqfrd{0}{p,\:q,\:r}{}{\bot}$}
        \RightLabel{$R\imp$}
        \UIC{$\seqfrd{0}{p,\:q,\:r}{}{\bot,\:\neg p}$}
        \DP
    \]
    ricordando che $\neg p=p\imp \bot$; notare che,
    $p\in\Clop{\{p,\:q,\:r\}}$ e $\bot\in\Clom{\{\bot\}}$. Per quanto riguarda il sequente $\s_{\eqref{ex1:3}}$, viene ottenuto applicando la regola $L\imp$ a $\s_{\eqref{ex1:2}}$:
    \[
        \AXC{$\seqfrd{0}{p,\:q,\:r}{}{\bot,\:\neg p}$}
        \RightLabel{$L\imp$}
        \UIC{$\seqfrd{0}{p,\:q,\:r}{}{\bot,\:\neg p,\:A}$}
        \DP
    \]
    dove $A=\neg(q\land r)=(q\land r)\imp \bot$ , notare che
    $\bot\in\Clom{\{\bot\}}$ e $q\land r\in\Clop{\{p,\:q,\:r\}}$.
    Il sequente $\s_{\eqref{ex1:5}}$ viene ottenuto applicando $\mathrm{Succ}$ a
    $\s_{\eqref{ex1:4}}$ spostando $r$ da sinistra a destra; similmente,
    $\s_{\eqref{ex1:7}}$ viene ottenuto applicando $\mathrm{Succ}$ a
    $\s_{\eqref{ex1:6}}$ spostando $p$ e $q$ da sinistra a destra e
    spostando $r$ nella zona pi\'u a destra. Abbiamo evidenziato con $*$ le premesse di $\mathrm{Succ}$ che contribuiranno alla costruzione del contromodello.
    Notare che il sequente $\s_{\eqref{ex1:11}}$ rispetta la propriet\'a $G\in\Clom{\D\cup\Lambda}$.
\end{example}

\begin{figure}[t]
    \centering
    \[
        \begin{array}{l}
            G \;=\;
            A\lor (p\imp r) \lor B \lor (C\imp (p\lor \neg p))
            \\[1.5ex]
            A\;=\,\neg(q\land r)
            \qquad
            B \;=\;(\neg \neg p \land (p\imp q))\imp q
            \qquad
            C\;=\;B\imp (\neg\neg p \land q)
            \\[1.5ex]
            SL^{At}{G}  \;=\;
            \{\: p,\:q,\:r\:\}
            \qquad
            SL^{At}{G}  \;=\;
            \{\:C,\: \neg \neg p,\: p\imp q\:\}
            \\[1.5ex]
            SR^{At}{G} \;=\;
            \{\: p,\:q,\:r \:\}
            \qquad
            SR^{At}{G} \;=\;
            \{\: A,\:p\imp r,\:B,\:C\imp(p\lor \neg p),\:\neg p\:\}
        \end{array}
    \]
    \leqnomode % print counters in the left
    \setcounter{equation}{0} % reset counter
    % \allowdisplaybreaks % allow page break inside align
    \begin{align}
        \label{ex1:1}
        \tseqfrd{0}{p,\:q,\:r}{}{\bot}
                                                    &  & \mathrm{Ax}
        \\[1ex]
        \label{ex1:2}
        \tseqfrd{0}{p,\:q,\:r}{}{\bot,\:\neg p}     &  & R\imp\; \eqref{ex1:1}
        \\[1ex]
        \label{ex1:3}
        \tseqfrd{0}{p,\:q,\:r}{}{\bot,\:\neg p,\:A} &  & R\imp\; \eqref{ex1:2}
        \\[1ex]
        \label{ex1:4}
        \tseqfrd{0}{\neg\neg p,\:p,\:q,\:r}{}{\bot,\:\neg p,\:A} \quad(*)
                                                    &  & L\imp\; \eqref{ex1:3}
        \\[1ex]
        % --------------------------------------------------------------------
        \hline
        \label{ex1:5}
        \tseqfrd{1}{\neg\neg p,\:p,\:q}{r}{\bot,\:\neg p,\:A}
                                                    &  & \mathrm{Succ}\; \eqref{ex1:4}
        \\[1ex]
        \label{ex1:6}
        \tseqfrd{1}{\neg\neg p,\:p,\:q}{r}{\bot,\:\neg p,\:A,\:p\imp r} \quad(*)
                                                    &  & R\imp\; \eqref{ex1:5}
        \\[1ex]
        % --------------------------------------------------------------------
        \hline
        \label{ex1:7}
        \tseqfrd{2}{\neg\neg p}{p,\:q} {\bot,\:\neg p,\:A,p\imp r,\:r}
                                                    &  & \mathrm{Succ}\; \eqref{ex1:6}
        \\[1ex]
        \label{ex1:8}
        \tseqfrd{2}{p\imp q,\:\neg\neg p}{p,\:q}{\bot,\:\neg p,\:A,p\imp r,\:r}
                                                    &  & L\imp\; \eqref{ex1:7}
        \\[1ex]
        \label{ex1:9}
        \tseqfrd{2}{p\imp q,\:\neg\neg p}{p,\:q}{\bot,\:\neg p,\:A,p\imp r,\:r,\:B}
                                                    &  & R\imp\; \eqref{ex1:8}
        \\[1ex]
        \label{ex1:10}
        \tseqfrd{2}{C,\:p\imp q,\:\neg\neg p}{p,\:q}  {\bot,\:\neg p,\:A,p\imp r,\:r,\:B}
                                                    &  & L\imp\; \eqref{ex1:9}
        \\[1ex]
        \label{ex1:11}
        \tseqfrd{2}{C,\:p\imp q,\:\neg\neg p}{p,\:q}
        {\bot,\:\neg p,\:A,p\imp r,\:r,\:B,\:C\imp (p\lor \neg p)}\quad(*)
                                                    &  & R\imp\; \eqref{ex1:10}
    \end{align}
    \caption{Costruzione del calcolo di refutazione ${\bf RGD}(G)$ di $G$; I p-sequenti sono evidenziati da (*).}
    \label{fig:ex1}
\end{figure}

\end{document}