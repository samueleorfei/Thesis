\providecommand{\main}{..}
\documentclass[\main/tesi.tex]{subfiles}
\begin{document}
\chapter{Conclusione}
\addcontentsline{toc}{chapter}{Conclusione}

Riassumendo, l'obiettivo del tirocinio è stato quello di produrre una implementazione di un algoritmo che potesse essere usato per la generazione di contro-modelli di formule non valide nella logica di G\"odel-Dummett. \\
Avendo già la totalità dei requisiti necessari, specifiche da seguire e regole da applicare, la progettazione è risultata lineare ed ha necessitato di poche correzioni durante lo sviluppo. Le correzioni sono state principalmente teoriche. \\
Per quanto riguarda l'implementazione, è stata posta molta attenzione alla modularità che ogni pezzo del progetto doveva avere, per poter, eventualmente, essere utilizzato anche in altri progetti, in maniera autonoma. \\

\section{Possibili sviluppi futuri}
In futuro la soluzione potrebbe ricevere svariati aggiornamenti riguardanti soprattutto l'implementazione di alcuni moduli.\\
Ad esempio, i moduli riguardanti l'interprete (Lexer e Parser) possono essere migliorati dal punto di vista del debugging, dell'ottimizzazione delle funzioni o nei riguardi della grammatica, che potrebbe essere ampliata per riconoscere più tipi di caratteri. \\
Oppure si potrebbero implementare le funzioni stesse che compongono l'algoritmo in maniera più ottimizzata. \\

\end{document}